\documentclass{mcmthesis}
\mcmsetup{
    CTeX = true,
    tcn = 10538,
    problem = F,
    sheet = true,
    titleinsheet = true,
    keywordsinsheet = true,
    titlepage = false,
    abstract = true
}

% ========== 段落缩进核心配置(优先级最高) ==========
\setlength{\parindent}{2em}       % 全局首行缩进2个汉字宽度
\setlength{\parskip}{0pt}         % 取消段间空行(中文排版规范)
\usepackage{indentfirst}          % 章节标题后第一段也缩进


% ========== 其他必要包 ==========
\usepackage{lipsum}  % 示例文本
\usepackage{amsmath} % 公式
\usepackage{graphicx}% 图片
\usepackage{listings}% 代码
\usepackage{tocloft} % 目录格式
\usepackage{setspace}% 行距(后续统一设置)
\usepackage{lastpage}% 总页数
\usepackage{fontspec}% 字体
\usepackage{hyperref}% 超链接(放最后加载)
\setmainfont{Times New Roman} % 英文主字体

% ========== 行距配置(可选,避免影响缩进) ==========
\onehalfspacing% 1.5倍行距(中文论文常用,可改为\doublespacing/ \singlespacing)

% ========== 代码列表格式 ==========
\lstset{
    basicstyle = \small\ttfamily,
    keywordstyle = \color{blue},
    commentstyle = \color{gray},
    stringstyle = \color{green},
    numbers = left,
    numberstyle = \tiny\color{gray},
    frame = tb,
    tabsize = 4,
    breaklines = true,
    breakatwhitespace = false
}

% ========== 标题/作者配置 ==========
\title{\textbf{Multi-dimensional Analysis: Identifying the Optimal Selection of Rental and Purchase Plans in Chongqing}}
\author{Team 10538} % 补充作者信息(数模竞赛通常写团队编号或成员姓名)



\begin{document}
% ========== 摘要(自动缩进,无需手动\indent) ==========
\begin{abstract}
\begingroup
\setlength{\parskip}{8pt} % 局部段距

With the increase in urbanization rate and the decrease in the number of births, the sales area of housing in many places has dropped significantly.
In this paper, we have adopted \textbf{multiple modeling schemes} and established a \textbf{multi-dimensional analysis system}, 
aiming to provide the public with more accurate and cost-effective housing reference plans, 
and offer reasonable suggestions to the Chongqing government
% 随着城市化率的提高和出生人口的减少,许多地方的住房销售面积明显下降。
% 在本文中,我们采用了多种建模方案,建立了多维度的分析体系,旨在为公众提供更准确、性价比更高的住房参考方案,
% 并为重庆市政府w提供合理的建议。

For task1, we build a \textbf{Analytic Hierarchy Process (AHP)} model.By combining the actual situation, 
establish A, B, and C as the target layer, criterion layer, and sub-criterion layer respectively, 
and set the corresponding scales. Based on the scales, write the judgment matrix,
and then determine the weights of the relevant influencing factors through the eigenvalue method.
% 对于任务1,我们建立了层次分析法(AHP)模型。
% 通过结合实际情况建立A,B,C为目标层,准则层,子准则层,并设定相应的标度,
% 依据标度写出判断矩阵,再通过特征值法来确定相关影响因素的权重

For task2, We approach the research from two core dimensions: 
\textbf{the triggering conditions for property replacement} and \textbf{the implementation pathways of property replacement.}
By constructing a simulation-based decision function, 
we quantify the weights of respective influencing factors and further deduce the necessity of property replacement.
% 我们从“住房置换的触发条件”与“住房置换的实施路径”这两个核心维度展开研究。
% 通过构建基于模拟的决策函数,我们对各影响因素的权重进行量化,并进一步推演住房置换的必要性。

For task3, we utilize \textbf{Grey Model (GM)} to forecast the housing price trends in Chongqing over the next five years.
% 对于任务3,我们利用灰色预测模型(GM)对重庆未来五年的房价走势进行预测。

For task4, We synthesized the findings from the aforementioned tasks and, 
based on the calculation results of the model, 
submitted a memorandum outlining project-specific measures to the Chongqing Municipal Government, 
with the aim of assisting the public in making more informed decisions regarding property rental and purchase.
% 我们综合了上述任务的结果,并根据模型的计算结果,
% 向重庆市政府提交了一份概述项目具体措施的备忘录,旨在帮助公众在物业租赁和购买方面做出更明智的决定。

\endgroup
\end{abstract}

\vspace{12pt}  % 摘要与关键词间距

\begin{keywords}
Analytic Hierarchy Process; Multiple Regression Analysis; Grey Model
\end{keywords}

\maketitle

% ========== 目录页 ==========
\newpage
\fancypagestyle{plain}{} 
\renewcommand{\contentsname}{\hfill Contents \hfill}
\setcounter{tocdepth}{3} 
\begin{spacing}{1.3}
    \tableofcontents
\end{spacing}
\setcounter{page}{2} 
\newpage

% ========== 正文(所有段落自动缩进) ==========
\section{Introduction}
Addison-Wesley Publishing Company, 1986.

\begin{itemize}
\item minimizes the discomfort to the hands, or
\item maximizes the outgoing velocity of the ball.
\end{itemize}

To model the collision between the ball and the bat, several physical quantities need to be considered:

\begin{itemize}
\item the initial velocity and rotation of the ball,
\item the initial velocity and rotation of the bat,
\item the relative position and orientation of the bat and ball, and
\item the force over time that the hitter hands applies on the handle.
\end{itemize}

\begin{itemize}
\item the angular velocity of the bat,
\item the velocity of the ball, and
\item the position of impact along the bat.
\end{itemize}

sales area of housing in many places has dropped significantly.

\begin{Theorem} \label{thm:latex}
\LaTeX
\end{Theorem}
\begin{Lemma} \label{thm:tex}
\TeX .
\end{Lemma}
\begin{proof}
The proof of theorem.
\end{proof}

\subsection{Other Assumptions}
\begin{itemize}
\item Assumption 1
\item Assumption 2
\item Assumption 3
\item Assumption 4
\end{itemize}

\lipsum[7]

\section{Analysis of the Problem}
\lipsum[8]
\begin{equation}
a^2 \tag{1} \label{aa}
\end{equation}

\[
\begin{pmatrix}*{3}c
a_{11} & a_{12} & a_{13} \\
a_{21} & a_{22} & a_{23} \\
a_{31} & a_{32} & a_{33} \\
\end{pmatrix}
= \frac{\text{Opposite}}{\text{Hypotenuse}} \cos^{-1}\theta \arcsin\theta
\]
\lipsum[9]

\[
p_{j}=
\begin{cases} 
0, & \text{if $j$ is odd} \\
r!\,(-1)^{j/2}, & \text{if $j$ is even}
\end{cases}
\]
\lipsum[10]

\[
\arcsin\theta =
\mathop{{\int\!\!\!\!\!\int\!\!\!\!\!\int}\mkern-31.2mu
\bigodot}\limits_\varphi
\mathop{\lim}\limits_{x \to \infty} \frac{n!}{r!\left(n - r\right)!}
\eqno (2)
\]

\section{Calculating and Simplifying the Model}
\lipsum[11]

\section{The Model Results}
\lipsum[6]

\section{Validating the Model}
\lipsum[9]

\section{Conclusions}
\lipsum[6]

\section{A Summary}
\lipsum[6]

\section{Evaluation of the Model}
\lipsum[12]

\section{Strengths and Weaknesses}
\lipsum[12]

\subsection{Strengths}
\begin{itemize}
\item \textbf{Applies widely}\\
This system can be used for many types of airplanes, and it also solves the interference during the procedure of boarding airplanes. As described above, we can optimize the boarding time. All services are automated.
\item \textbf{Improve the quality of airport service}\\
Balancing costs and benefits, it brings more convenience to airports and passengers, and saves human resources for airlines.
\item \textbf{High reliability}\\
The model is validated through multiple scenarios, ensuring stable performance in practical applications.
\end{itemize}

\begin{thebibliography}{99}
\bibitem{1} 中国经济网, 符仲明, & 符仲明. (n.d.). 王萍萍:人口总量有所下降 人口高质量发展取得成效_中国经济网——国家经济门户. \urlh{ttp://www.ce.cn/xwzx/gnsz/gdxw/202401/18/t20240118_38870849.shtml}
\bibitem{2} Leslie Lamport. \textit{\LaTeX{}: A Document Preparation System}. Addison-Wesley Publishing Company, 1986.
\bibitem{3} \url{http://www.latexstudio.net/}
\bibitem{4} \url{http://www.chinatex.org/}
\end{thebibliography}

\begin{appendices}
\section{First Appendix}
\lipsum[13]
Here are the simulation programs used in our model:

\textbf{\textcolor[rgb]{0.98,0.00,0.00}{Matlab Source Code:}}
% \lstinputlisting[language=Matlab]{./code/mcmthesis-matlab1.m}

\section{Second Appendix}
Some more text. \textbf{\textcolor[rgb]{0.98,0.00,0.00}{C++ Source Code:}}
% \lstinputlisting[language=C++]{./code/mcmthesis-sudoku.cpp}
\end{appendices}
\end{document}